% !TEX TS-program = XeLaTeX
% Set the font to a nice monospaced one
% \setmainfont{Courier}
\setmainfont[
    Path=./fonts/Lato/,
    Extension = .ttf,
    UprightFont = *-Regular,
    BoldFont = *-Bold
]{Lato}
\setmonofont[
    Path=./fonts/Courier_Prime/,
    Extension = .ttf,
    UprightFont = *-Regular,
    BoldFont = *-Bold,
    ItalicFont = *-Italic
]{CourierPrime}

% supplemental information (lighter-colored text)
\definecolor{supplemental}{HTML}{C0C0C0}
% #EDBD7D Background
\definecolor{background}{HTML}{EDBD7D}
% #2D7539 Light infantry - green heart
\definecolor{infantry}{HTML}{2D7539}
% #F9B330 Light vehicle - yellow heart
\definecolor{light-vehicle}{HTML}{F9B330}
% #CC1E29 Heavy vehicle - red heart
\definecolor{heavy-vehicle}{HTML}{CC1E29}
% #5D2362 Emplacement - purple heart
\definecolor{emplacement}{HTML}{5D2362}

% Use fbox, but don't draw borders
% This helps add a little buffer space between elements
% I comment this out if I'm adjusting where components are so I can see the borders
\setlength{\fboxrule}{0pt}
